\documentclass[11pt]{article}
\usepackage[T1]{fontenc}
\usepackage{lmodern}
\usepackage{microtype}
\usepackage{amsmath, amssymb}
\usepackage{geometry}
\geometry{margin=1in}
\usepackage{hyperref}
\hypersetup{colorlinks=true, linkcolor=blue, urlcolor=blue, citecolor=blue}
\usepackage{listings}
\lstset{basicstyle=\ttfamily\small, breaklines=true}

\title{Background on Tensor Contraction Kernels in Uni20}
\author{Uni20 Developer Documentation}
\date{\today}

\begin{document}

\maketitle

\section{Introduction}
Tensor contraction is the multilinear generalisation of matrix multiplication.  Uni20 exposes
contraction kernels that operate on tensor operands described by \texttt{mdspan}-compatible
interfaces.  This note collects the background theory and implementation strategy used by Uni20's
current kernels so that new contributors can reason about correctness and performance trade-offs.
We focus on the CPU and BLAS backends, where the core strategy is to decompose an arbitrary
contraction into groups of indices that resemble the \((M,N,K)\) structure of a GEMM operation.

\section{Terminology and Layout Conventions}\label{sec:terminology}
The C++ \texttt{mdspan} reference implementation distinguishes between
\texttt{layout\_right}---where the right-most index varies fastest and therefore has stride~1---and
\texttt{layout\_left}, whose left-most index is contiguous.  Uni20 adopts this vocabulary verbatim
for concrete layouts while reserving the shorter ``right'' and ``left'' adjectives for the
corresponding traversal heuristics.  When we describe a heuristic as ``right-biased'', we mean that
it follows the same prioritisation of contiguous axes as \texttt{layout\_right}; likewise a
``left-biased'' heuristic mirrors \texttt{layout\_left}.  Later discussions of
\texttt{merge\_strides\_right()} and \texttt{merge\_strides\_left()} (Section~\ref{sec:merge-order})
use this convention to indicate which end of the descriptor list they prefer when ordering strides.

Throughout the remainder of this document we use \emph{merging} as the canonical term for combining
adjacent stride descriptors whose outer stride equals the inner stride multiplied by the extent.
This replaces the earlier ``coalescing'' wording and aligns the prose with the
\texttt{merge\_strides\_*} helper names.  Forward references to the merging algorithms in
Section~\ref{sec:merge-order} establish the terminology before the algorithms themselves appear.

\section{From \texttt{mdspan} to Concrete Tensors}
Uni20 adopts the C++ reference implementation of \texttt{mdspan} (distributed under the
\texttt{std::experimental} namespace) but follows the convention of importing it into the
\texttt{stdex} namespace, so within this project the core view type is \texttt{stdex::mdspan}.  An
\texttt{mdspan} couples a
raw pointer (``data handle'') with an \emph{extents} object, a layout mapping, and an accessor.
Within Uni20 we wrap that interface in convenience aliases and concepts (see
\texttt{src/common/mdspan.hpp} and \texttt{src/mdspan/concepts.hpp}).  Any type that models the
\texttt{StridedMdspan} concept exposes the rank, extents, and per-index strides required to navigate
strided storage.  The kernel front-end therefore accepts operands as \texttt{StridedMdspan}
instances, enabling it to operate on plain \texttt{mdspan}, custom views, and tensor wrappers alike.

A concrete owning tensor pairs storage with such an \texttt{mdspan}.  The class template
\texttt{Tensor<\ldots>} stores an owning buffer and an \texttt{mdspan} view over that buffer
(Fig.~\ref{fig:tensor-mdspan}).  The kernel path only interacts with the non-owning view: when a
tensor is passed to a contraction routine we immediately retrieve its \texttt{mdspan} via
\texttt{view()} and use \texttt{data\_handle()}, extents, and strides to drive the computation.
This makes the implementation agnostic to how memory is allocated while keeping the backend code
focused on pointer arithmetic.

\begin{figure}[h]
  \centering
  \begin{minipage}{0.9\linewidth}
    \begin{lstlisting}
class Tensor {
  using mdspan_type = stdex::mdspan<T, extents_type, layout_type, accessor_type>;
  std::vector<T> storage_;   // owning buffer
  mdspan_type    view_;      // mdspan over that buffer
};
    \end{lstlisting}
  \end{minipage}
  \caption{A Uni20 tensor owns storage but exposes an \texttt{mdspan} view.  Backend kernels consume
  the \texttt{mdspan} interface only.}
  \label{fig:tensor-mdspan}
\end{figure}

\section{Stride Group Extraction}
Consider a contraction of rank-$A$ tensor $A$ and rank-$B$ tensor $B$ over $N$ contracted index
pairs.  Uni20 reorganises the indices into three groups, each represented by a small array of
\texttt{extent\_strides} structures:
\begin{itemize}
  \item The \emph{$M$-group} stores all uncontracted indices sourced from $A$ (which also appear in
        the output tensor $C$).
  \item The \emph{$N$-group} stores all uncontracted indices sourced from $B$ (also present in $C$).
  \item The \emph{$K$-group} contains the contracted index pairs between $A$ and $B$.
\end{itemize}
Each descriptor captures the extent of the grouped dimension and the strides in both participating
operands.  The helper \texttt{extract\_strides()} walks the contracted index pairs, checks that
extents agree, and fills the $K$-group with matching stride metadata.  It then visits the remaining
indices of $A$ and $B$, checking that they match the layout of $C$ while filling the $M$- and
$N$-groups respectively.  Finally, each group is merged: adjacent descriptors combine when both
operands present mergeable strides (the outer stride equals the inner stride scaled by the extent),
effectively fusing neighbour indices that already form a single logical block.  Uni20's implementation
calls \texttt{merge\_strides\_right()} on each group, yielding a compact representation of the
contraction that resembles GEMM's $(M,N,K)$ tuple.

\section{Ordering Behaviour of Stride Utilities}\label{sec:merge-order}
The descriptors returned by \texttt{extract\_strides()} are appended in a fixed logical order:
contracted pairs populate the $K$ group first, followed by the remaining legs of $A$ and $B$ that
populate the $M$ and $N$ groups alongside their matching indices in $C$.  When the resulting
\texttt{static\_vector} instances are passed to the in-place overload of
\texttt{merge\_strides\_right()} they are only compacted by merging adjacent entries while
preserving this logical ordering.  Readers who need a reminder of the terminology behind $M$, $N$, and
"merging" can revisit Section~\ref{sec:terminology}.

Helpers that construct iteration plans may invoke either \texttt{merge\_strides\_right()} or
\texttt{merge\_strides\_left()} depending on the desired traversal order.  The ``right'' variant
implements the same prioritisation of contiguous axes as \texttt{layout\_right} described in
Section~\ref{sec:terminology}: descriptors are sorted by the absolute value of their primary stride
in descending order so that row-major friendly (right-biased) traversals merge outer indices before
inner ones.  The ``left'' variant mirrors the column-major semantics of \texttt{layout\_left} by
sorting those absolute strides ascending, allowing left-biased traversals to favour merges that keep
the leading logical axis contiguous.  Both helpers therefore expose the layout-driven semantics of
the earlier section while operating on stride descriptors rather than concrete views.

Regardless of ordering, the helpers only perform a merge when the mergeability predicate holds in the
merging sense: the outer stride of the candidate descriptor must equal the inner stride multiplied by
the extent for every operand participating in the comparison.  This restatement ensures the
terminology aligns with the merging contract enforced throughout Section~\ref{sec:terminology}.

The ordering choice communicates intent to the contraction planner.  Right-biased merges are a better
fit when we expect the final kernel to execute in a row-major friendly fashion---for example, when
packing into buffers that expect \texttt{layout\_right}-style contiguity or when iterating directly
over tensors stored with stride~1 on the trailing axis.  Conversely, left-biased merges should be
preferred for column-major friendly plans, such as when the output will flow into BLAS routines that
assume \texttt{layout\_left}-style contiguity or when the leading axis must remain contiguous for
cache locality.

This split also surfaces where additional heuristics could help.  Sorting solely on the primary
stride fails to notice when the chosen operand cannot be merged but a different tensor could.  A
more sophisticated approach could fall back to the second stride (or additional tie-break criteria)
once it detects that the primary ordering prevents merges, or even allow callers to select the
preferred tensor when only one operand is scheduled for rearrangement.  Such changes would make
the heuristic explicit and would simplify contraction decisions in cases where we wish to keep a
particular tensor untouched while packing the others.

\section{CPU Reference Kernel}
The CPU reference backend executes the grouped contraction via recursive loops.  The $M$-group is
iterated outer-most, the $N$-group governs the middle loops, and the $K$-group contributes to the
innermost dot product.  Each recursion level adjusts the active pointers by the precomputed strides
so that no additional index arithmetic is required.  This implementation serves two purposes: it is
an always-available fallback and, more importantly, it codifies the semantics used to validate more
specialised backends.  Because the recursion operates directly on \texttt{mdspan}-derived strides,
any layout that satisfies \texttt{StridedMdspan} can be contracted without additional preparation.

\section{Delegating to BLAS}
When the grouped representation aligns with the requirements of a GEMM, the contraction can be
mapped to a single BLAS call:
\begin{equation}
  C_{MN} \gets \beta C_{MN} + \alpha A_{MK} B_{KN}.
\end{equation}
A BLAS GEMM operates on column-major matrices with the following key properties:
\begin{itemize}
  \item Each operand must present at least one axis with unit stride.  The $\texttt{TRANS}$
        parameter selects whether that axis corresponds to rows or columns, effectively allowing
        either row-major or column-major interpretation for each input matrix.
  \item The leading dimensions (strides between consecutive columns/rows) may exceed the logical
        extents, allowing batched or padded storage.
  \item The output matrix $C$ likewise requires a unit-stride axis, but by choosing the transpose
        flags consistently we can always align the stored unit stride with BLAS's expectation.  When
        $C$ itself is row-major we can compute $C^\top = B^\top A^\top$ instead and transpose the
        interpretation of every operand simultaneously.
\end{itemize}
Uni20's stride descriptors already capture both extents and strides.  A direct GEMM is therefore
possible when:
\begin{enumerate}
  \item Each of the $M$-, $N$-, and $K$-groups collapses to a single descriptor so that every
        operand presents exactly two logical axes.
  \item For $A$ either the $M$- or the $K$-descriptor has stride $1$.  The transpose flag determines
        which logical axis BLAS treats as contiguous.
  \item For $B$ either the $K$- or the $N$-descriptor has stride $1$, again selecting the appropriate
        transpose mode.
  \item For $C$ either the $M$- or the $N$-descriptor has stride $1$.  If $C$ is row-major we flip
        both operands so that BLAS computes $C^\top = B^\top A^\top$ and then reinterpret the result.
\end{enumerate}
Because BLAS exposes two transpose choices per operand there are $2\times2=4$ combinations for the
inputs and, once the result orientation is taken into account, eight total possibilities when
deciding which physical axis supplies the unit stride.  When these criteria hold, the grouped
contraction represents a simple matrix multiplication with
matrices laid out in memory exactly as BLAS expects.  Uni20 forwards the contraction to the BLAS
backend, providing extents as matrix sizes and the outer strides as leading dimensions.

Although the logical requirements above are symmetric, practical implementations of GEMM usually
achieve their highest throughput when the contracting ($K$) group is contiguous in both operands.
In column-major Fortran notation this corresponds to computing $C \leftarrow A^\top B$, so that the
transpose flag exposes the $K$-group as the unit-stride axis in $A$ while $B$ already presents a unit
stride along its rows.  Modern BLAS libraries pack input panels into contiguous buffers, which
reduces the penalty when one operand lacks stride-1 access; nevertheless, presenting contiguous
memory along $K$ minimises packing cost and aligns best with vendor-optimised micro-kernels.

\section{Rearrangement Strategies}
Many practical contractions violate at least one of the BLAS-friendly criteria.  The two common
issues are:
\begin{enumerate}
  \item \textbf{Grouped indices that are not contiguous.}  After merging we may still retain
        multiple descriptors per group because strides disagree.  For instance, an $M$-group may
        contain two indices where the stride of the inner index is not the extent-scaled stride of
        the outer index.  This prevents the fused dimension from being interpreted as a single
        matrix axis.
  \item \textbf{Operand has no stride-1 index.}  If a tensor has no stride-1 index (for example,
        it might be obtained by \emph{slicing} a tensor, e.g. taking every second element of a
        tensor\footnote{A practical example of this is taking the real components of a complex
        tensor.}) then that tensor is not compatible with BLAS.
\end{enumerate}
To exploit BLAS in these scenarios we must rearrange storage so that each group behaves like a
single contiguous axis.  Uni20 can deploy several strategies:
\begin{description}
  \item[Looped GEMM.]  If only one group fails to merge completely, we can treat the remaining
        dimensions as batch loops around a GEMM call.  For example, suppose the $M$-group contains
        two descriptors but the $N$- and $K$-groups each reduce to one.  We can iterate over the
        extra $M$-group descriptor, adjusting pointers by its stride, and issue a GEMM for every
        slice.  This preserves GEMM's efficiency while avoiding a full rearrangement.
  \item[Rearrange $\rightarrow$ GEMM $\rightarrow$ Rearrange.]  When multiple groups have
        non-contiguous descriptors or when unit stride axes do not line up, we may materialise
        temporary buffers whose layout is explicitly column-major.  The sequence proceeds as follows:
        \begin{enumerate}
          \item Copy or pack the relevant operand(s) into temporaries with contiguous grouped axes.
          \item Call GEMM on the packed operands.
          \item Unpack or scatter the result back to the original layout of $C$.
        \end{enumerate}
        The packing step often leverages the same stride metadata to generate efficient nested loops
        or to drive vectorised copy kernels.
\end{description}

In practice Uni20 performs these rearrangements by materialising \texttt{mdspan} views over the
temporary buffers and reusing the stride descriptors gathered during grouping.  Each descriptor
provides an extent and two strides; packing iterates over the logical extents in lexicographic order,
computes the source offset via the strides, and writes into a contiguous destination \texttt{mdspan}
whose layout matches the intended GEMM operand.  The unpacking phase mirrors this process for $C$.
Because both stages operate exclusively through \texttt{mdspan}-compatible interfaces, they respect
custom accessors and allow future backends to substitute specialised copy kernels when available.

The choice between these approaches depends on the relative size of the non-contiguous groups,
cache behaviour, and the overhead of packing.  For high-order tensors where several groups resist
merging, a full rearrangement can be amortised across the large volume of arithmetic in the GEMM.
Conversely, when only a small extent prevents direct GEMM dispatch, keeping the native layout and
looping over GEMM calls typically wins.

\section{Putting It All Together}
A general tensor contraction can therefore follow one of several execution paths:
\begin{enumerate}
  \item \textbf{Direct GEMM.}  When every group collapses to a single contiguous descriptor with
        unit stride in the contracting operands, Uni20 maps the operation to a single BLAS GEMM.
  \item \textbf{Batched GEMM.}  When a subset of groups contains descriptors that could not be merged
        but the remaining ones do merge, Uni20 can run GEMM inside outer loops indexed by the residual
        descriptors.
  \item \textbf{Pack--GEMM--Unpack.}  When neither of the above is feasible, Uni20 can rearrange the
        operands and/or result into contiguous temporaries, dispatch GEMM, and scatter the result
        back.
  \item \textbf{Reference Loop Kernel.}  As a fallback or correctness reference, the CPU recursive
        kernel executes the contraction directly from the stride descriptors without relying on
        GEMM.  This path also supports layouts that cannot be cheaply rearranged.
\end{enumerate}
These execution modes all stem from the same stride grouping produced by
\texttt{extract\_strides()}.  By reasoning in terms of \texttt{mdspan} extents and strides, Uni20 is
able to describe both generic and highly optimised kernels within a unified abstraction.

\section{Worked Example}
To illustrate the mechanics of stride grouping and rearrangement, consider the contraction
\begin{equation}
  C_{aij} = \sum_b A_{abi} B_{bj}
\end{equation}
with the following layouts (extents and per-index strides):
\begin{align*}
  C &\in \mathbb{R}^{3 \times 4 \times 5}, &\text{strides}&=(20, 5, 1), \\
  A &\in \mathbb{R}^{3 \times 7 \times 4}, &\text{strides}&=(28, 4, 1), \\
  B &\in \mathbb{R}^{7 \times 5},           &\text{strides}&=(5, 1).
\end{align*}
The index order in the explicit expression is not semantically relevant; only the mapping between
indices and tensor layouts matters when extracting stride groups.  In particular, the permutation
\(C_{iaj}\) with extents \((4, 3, 5)\) and strides \((5, 20, 1)\) is equivalent from the
contraction engine's perspective.

\paragraph{Initial stride grouping.}  Running \texttt{extract\_strides()} identifies two potential
assignments for the $M$-group (the uncontracted indices sourced from $A$):
\begin{itemize}
  \item $M = (a, i)$ with extents $(3, 4)$, strides $(28, 1)$ in $A$, and corresponding strides
        $(20, 5)$ in $C$.
  \item $M = (i, a)$ with extents $(4, 3)$, strides $(4, 28)$ in $A$, and strides $(5, 20)$ in $C$.
\end{itemize}
In both cases the $K$-group contains the single contracted index $b$ with extent $7$, stride $4$ in
$A$, and stride $5$ in $B$.  The $N$-group consists of the index $j$ with extent $5$, stride $1$ in
$B$, and stride $1$ in $C$.  Either $M$-group choice preserves the contraction semantics, but neither
configuration yields a stride-$1$ $K$-group in $A$, preventing a direct BLAS dispatch without further
processing.

\paragraph{Rearranging operand $A$.}  The kernel therefore decides to pack $A$ into a temporary
layout where the $K$-group is contiguous.  One suitable rearrangement stores the indices in the order
$(b, i, a)$ so that the packed tensor has strides $(1, 7, 28)$: the $b$ index (contracting group)
occupies unit stride, while $(i, a)$ form the $M$-group with extents $(4, 3)$ and strides $(7, 28)$.
After merging, the grouped descriptors become
\begin{align*}
  M &: \text{extent } 12, &\text{stride}_A &= 7, &\text{stride}_C &= 5, \\
  N &: \text{extent } 5,  &\text{stride}_B &= 1, &\text{stride}_C &= 1, \\
  K &: \text{extent } 7,  &\text{stride}_A &= 1, &\text{stride}_B &= 5,
\end{align*}
which fits the GEMM interface with $C_{MN} = A_{MK} B_{KN}$.  Because $K$ is now contiguous in both
operands, the BLAS call operates without additional packing overhead.

\paragraph{Matching result layout.}  The choice of permutation during packing matters when deciding
whether $C$ also requires rearrangement.  If $C$ already presents a mergeable $M$-group---as in
this example with strides $(20, 5)$---the kernel can align the packed $A$ with the result layout and
avoid touching $C$ beyond the GEMM itself.  Conversely, had $C$ used non-mergeable strides (for
instance $(5, 15, 1)$) the kernel could pack $A$ with strides $(7, 1, 21)$ so that the fused $M$
dimension matches $C$'s memory order, allowing the GEMM output to be written in-place without a
separate scatter.

\paragraph{Decision test.}  The rule of thumb is:
\begin{enumerate}
  \item Check whether the $K$-group in $C$ is mergeable (contiguous after merging strides).  If it
        is, choose the permutation of packed strides in $A$ that mirrors $C$'s grouping so that no
        additional work on $C$ is required.
  \item If the $K$-group is not mergeable in $C$, $C$ must be rearranged regardless.  In that case the
        packed layout for $A$ (and subsequently $C$) can be chosen freely because both operands will
        be materialised into temporaries.
\end{enumerate}
This worked example demonstrates how the abstract stride descriptors produced by
\texttt{extract\_strides()} drive concrete decisions about packing order, BLAS dispatch, and optional
rearrangement of the output tensor.

\end{document}